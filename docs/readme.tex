\documentclass[a4paper,30pt]{article}

\usepackage{graphicx} % Required for inserting images
\usepackage[english,russian]{babel}
\usepackage[utf8]{inputenc}
\usepackage{hyperref}
\usepackage{color}
\usepackage{listings}
\usepackage[%
    left=0.50in,%
    right=0.50in,%
    top=1.0in,%
    bottom=1.0in,%
    paperheight=11in,%
    paperwidth=8.5in%
]{geometry}%



\title{\huge \bfЛабораторная работа по \LaTeX}
\author{\textit{ Шумилин Дмитрий Степанович М3116}}
\date{Октябрь 2024}

\begin{document}
\maketitle

    \begin{flushright} \large
    \emph{Преподаватель:}\\
    \quad\quad\quadХасан \textsc{К.~А.}
    \end{flushright}

\newpage
\begin{center}
    \section*{\huge Документация для \href{https://github.com/smartiqaorg/geometric_lib}{geometric lib}}
\end{center}
\\
\begin{center}
\Large    
Библиотека geometric\_lib предназначена для упрощения подсчета площади и периметра некоторых фигур таких как:

\begin{itemize}
    \item Круг
    \item Квадрат
    \item Треугольник
    
\end{itemize}
\end{center}
\\
\section*{calculate.py}
\lstset{
language=Python,
basicstyle=\sffamily,
emph={import,def,return,while,if},
emphstyle=\color{red},
emph={[2]area,perimeter,calc,print,input,eval,in,not},
emphstyle=[2]\color{blue},
emph={[3]list,len,map,int},
emphstyle=[3]\color{cyan},
numbers=left,
commentstyle=\color{green},
numberstyle=\color{gray},
stepnumber=1,
numbersep=5pt,
backgroundcolor=\color{white},
escapeinside={\%#}{*)},
breaklines=true
}

\begin{lstlisting}
import circle
import square

figs = ['circle', 'square'] 
funcs = ['perimeter', 'area'] 
sizes = {} 

def calc(fig, func, size): 
  assert fig in figs 
  assert func in funcs

  result = eval(f'{fig}.{func}(*{size})') 
  print(f'{func}of{fig}is{result}') 

if __name__ == "__main__":
  func = ''
  fig = ''
  size = list()
    
  while fig not in figs:
    fig = input(f"Enter figure name, avaliable are {figs}:\n")
  
  while func not in funcs:
    func = input(f"Enter function name, avaliable are {funcs}:\n")
  
  while len(size) != sizes.get(f"{func}-{fig}", 1):
    size = list(map(int, input("Input figure sizes separated by space, 1 for circle and square\n").split(' ')))
  
  calc(fig, func, size)
\end{lstlisting}
\\
\begin{center}
\largeДанная программа позволяет считать по вводимым данным площадь и периметр всех доступных фигур
\end{center}
\newpage
\section*{circle.py}

\begin{lstlisting}
import math

def area(r):
    return math.pi * r * r

def perimeter(r):
    return 2 * math.pi * r


\end{lstlisting}

\begin{center}
\largeДанная программа принимает значение радиуса и в зависимости от операции использует определенную формулу. 
Для площади: $\pi r^2$ , а для периметра: $2\pi r$.      
\end{center}



\section*{square.py}

\begin{lstlisting}
def area(a):
    return a * a

def perimeter(a):
    return 4 * a
\end{lstlisting}
\begin{center}
\large 
Данная программа принимает значение стороны квадрата и может выполнять 2 операции:
\begin{enumerate}
    \item Вычисление площади: $a^2$
    \item Вычисление периметра: $4a$
\end{enumerate}
\end{center}

\section*{triangle.py}
\begin{lstlisting}
def area(a, b, c):
    return (a + b + c) / 2

def perimeter(a, b, c):
    return a + b + c
\end{lstlisting}
\begin{center}
\large
Данная программа принимает значение сторон треугольника и может выполнять 2 операции:
\begin{enumerate}
    \item Вычисление площади: $\frac{(a+b+c)}{2}$
    \item Вычисление периметра: $a+b+c$
\end{enumerate}
\end{center}

\section*{Ссылки:}
-\href{https://ru.overleaf.com/read/pcqnnhntbbtn#2bfdaf}{\textbf{Overleaf}}
\\
-\href{https://github.com/dshumilin03/geometric_lib/blob/documentation-latex/docs/readme.tex}{\textbf{GitHub}}
\end{document}
